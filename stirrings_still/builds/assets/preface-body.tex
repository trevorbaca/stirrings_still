\textbf{Scordatura.} The violins are tuned as usual; string IV of the viola is tuned down
a major second to B$\flat$2; string IV of the cello is tuned down a major second to
B$\flat$1:

\begin{tabu}{l l l}
\phantom{M} & violins: & E5, A4, D4, G3 \\
            & viola: & A4, D4, G3, B$\flat$2 \\
            & cello: & A3, D3, G2, B$\flat$1 \\
\end{tabu}

The score is written at sounding pitch.

\textbf{Accidentals.} Accidentals govern only one note, though natural signs appear as
reminders thoughout.

\textbf{String contact points (SCPs).} Three primary string contact points appear in the
score: tasto (\textbf{T}), ordinario (\textbf{O}), ponticello (\textbf{P}). Intepret
these musically: treat \textbf{T} as an invitation to find veiled, aerial colors; treat
\textbf{P} as an invitation to a find bright, possibily acidic colors; let musical
context be the guide. Do not play every \textbf{T} (or \textbf{P}) the same: the
\textbf{T} at one moment may be played considerably further up the fingerboard than the
\textbf{T} that follows. This is especially important in chained transitions between
contact points (\textbf{T} $\longrightarrow$ \textbf{P} $\longrightarrow$ \textbf{T}): do
not treat these as alternations between \textit{the same two values} of tasto and
ponticello over and over again, but rather as invitations to use color to work with ---
or against --- changes in dynamics, tempo and pitch at different moments in the piece.
(Measures 593\,--\,608 notate increasing degrees of tasto, from \textbf{T1} to
\textbf{T2} to \textbf{T3} to \textbf{T} \textit{poss.} The notation appears only at this
moment, structured as something of an elaboration of the color between the players.)

\textbf{Bow contact points (BCPs).} Fractions above the staff give contact points that
specify which part of the bow is to touch the string at which time. BCPs are given as
fractions between 0 and 1, with 0 indicating the talon and 1 indicating the tip;
intermedial values indicate a point somewhere between. For example, the BCPs $\frac{0}{4}
\longrightarrow \frac{1}{4} \longrightarrow \frac{2}{4}$ indicate a stroke drawn smoothly
from the talon ($\frac{0}{4}$), to a point one quarter of the way up the bow
($\frac{1}{4}$), and then to exactly the midway point of the bow ($\frac{2}{4}$); the
BCPs $\frac{6}{7} \longrightarrow \frac{7}{7} \longrightarrow \frac{0}{7}$ indicate a
stroke drawn smoothly from $\frac{6}{7}$ of the way up the bow, up to the tip of the bow
($\frac{7}{7}$), and then all the way back to the talon ($\frac{0}{7}$). Almost all the
bow contact points in the piece derive from a repeating pattern of 15 bow changes. All
points divide the bow into four ($\frac{0}{4}, \frac{1}{4}, \frac{2}{4}, \frac{3}{4},
\frac{4}{4}$) or seven ($\frac{0}{7}, \frac{1}{7}, \frac{2}{7}, \frac{3}{7}, \frac{4}{7},
\frac{5}{7}, \frac{6}{7}, \frac{7}{7}$) parts; the BCPs $\frac{0}{4}$ and $\frac{0}{7}$
are synonyms for the talon; the BCPs $\frac{4}{4}$ and $\frac{7}{7}$ are synonyms for the
tip. The bow contact points in \textit{Stirrings Still} animate pitched lines (frequently
glissandi) in two ways at the same time: first, with structured --- and sometimes quite
fast --- changes of bow; and, second, with frequently surprising changes of bow color,
from scratch to flautando and back again. This happens because all the BCPs in the score
are rhythmed: transitions between contact points drive changes in the \textit{speed} of
the bow along its course. This means that overly exacting interpretations of, say,
$\frac{5}{6}$ versus $\frac{6}{6}$ aren't necessary to bring out the constant changes of
color the notation invites.

\textbf{Circular bowing.} The piece makes extensive use of circular bowing, always
introduced by an arrowed circle and a directive giving the width (or degree of stricture)
of the circled stroke: \textit{tight}, \textit{very tight}, \textit{wide}, \textit{very
wide}, and so on. (The location of the stroke on the string --- whether closer to the
bridge or the nut --- is left to the players; the priority is a kaleidoscopic variegation
of color.) \textbf{Stems indicate the rate at which circular strokes repeat.} So, the
first appearance of the technique, at measure 23, specifies four (moderately wide)
circular strokes in the first violin, against three in the second violin, against five in
the viola. The combined ``circular polyrhythm'' is crucial to convey: think the turning
of gears in a fantastical machine that we hear but do not see. Note that these rhythms
layer different bowspeed \textit{colors} on top of one another, rather than different
accent chains as in conventional polyrhythm: because the viola's five circles are faster
than the violins' (and because all three instruments are playing circles of roughly the
same width) the viola's color is lighter and faster (and closer to a type of circular
flautando) than of either of the violins'. (Resulting differences of color are more
pronounced in the cello versus the other instruments: the cello's \textit{wide} stroke
travels a greater distance than the other instruments', giving the cello a ``faster
color'' than the other instruments at the same width.)

\textbf{ Damping, and 1- \& 2-finger harmonics.} The harp-damping symbol indicates that
the string should be damped by laying three fingers lightly (``with harmonic pressure'')
across it. Natural (``1-finger'') harmonics are played in the usual way. But notes marked
with a pair of open diamonds are special to this score: play these ``2-finger'' harmonics
by lightly placing a first finger at the position of the notated pitch and then laying
down a second finger (with the same pressure) just behind the first, i.e., between the
first finger and the nut. The effect of the 2-finger harmonics is beautiful, and
uncertain: the perceived pitch is related to both the natural harmonic (with 1
finger) and the damped fundamental (with 3 fingers), though how is unclear. Allow the
sounds of the 2-finger harmonics to be what they are.

\textbf{XFB.} The indication XFB stands for ``extreme fast bow'' flautando and indicates
a type of very fast, very light, rhythmically desynchronized skimming of the bow across
the string. The technique was introduced in \textit{Akasha} and should be played as in
the previous piece.

\textbf{Tremolo.} The modifiers \textit{largho} (wide) and \textit{stretto} (tight) are
used to indicate rates of tremolo that relatively slower and faster than normal. All
tremolo are rhythmically irregular.

\textbf{Bowing à la clarinette basse.} The indication 2$^\circ$ indicates that the bow
should be positioned at exactly the halfway point between bridge and the location of the
stopped string, i.e., at the location of the second partial of the sounding pitch. Though
the position takes a moment to find, bowing at the location of the second partial has the
effect of collapsing the spectral content of the even-numberered partials in favor of the
odd-, analogously to the sound of the bass clarinet. (The ``rasp'' directive that
frequently appears together with the technique asks for a type of slow bow at the
beginning of the stroke so that the process of beginning-to-speak is heard at the start
of the sound.)

\textbf{Fermatas.} The score notates four types of fermata: short, medium, long, very
long. These are the same as \textit{Akasha} and should be interpreted as in the previous
piece: the short fermata should be surprisingly short, suggesting the time of only a
fraction of beat; the medium fermata should be interpreted as a usual fermata; the long
fermata should last long enough that our memory begins to retrace what was last heard;
and the very long fermata should last long enough that we \textit{almost} begin to forget
the previous flow of musical time. (Fermatas over the narrator's text document decisions
about the distribution of the text but aren't to be performed.)

\textbf{Miscellaneous.} Flat ``glissandi'' are ties. Do not rearticulate the note heads
the separate segments of true glissandi. The symbol $\dashv$ means ``stop-on-string'':
deaden the string with the bow at the end of the notated duration.
