\textbf{Scordatura.} The violins are tuned as usual; string IV of the viola is tuned down
a major second to B$\flat$2; string IV of the cello is tuned down a major second to
B$\flat$1:

\begin{tabu}{l l l}
\phantom{M} & violins: & E5, A4, D4, G3 \\
            & viola: & A4, D4, G3, B$\flat$2 \\
            & cello: & A3, D3, G2, B$\flat$1 \\
\end{tabu}

The score is written at sounding pitch.

\textbf{Accidentals.} Accidentals govern only one note, though natural signs appear as
reminders thoughout.

\textbf{String contact points (SCPs).} Three primary string contact points appear in the
score: tasto (\textbf{T}), ordinario (\textbf{O}), ponticello (\textbf{P}). Intepret
these indications musically: treat \textbf{T} as an invitation to find veiled, aerial
colors; treat \textbf{P} as an invitation to a find bright, possibily acidic colors; let
musical context guide the search for these colors. Do not play every \textbf{T} (or
\textbf{P}) the same: the \textbf{T} at one moment may be played considerably further up
the fingerboard than the \textbf{T} that follows. This is especially important in chained
transitions between contact points (\textbf{T} $\longrightarrow$ \textbf{P}
$\longrightarrow$ \textbf{T}): do not treat these as alternations between \textit{the
same two values} of tasto and ponticello over and over again, but rather as invitations
to use color to work with --- or against --- changes in dynamics, tempo and pitch at
different moments in the piece. (Measures 593\,--\,608 notate increasing degrees of
tasto, from \textbf{T1} to \textbf{T2} to \textbf{T3} to \textbf{T} \textit{poss.} The
notation appears only at this moment, structured as something of an elaboration of the
color between the players.)

\textbf{Bow contact points (BCPs).} Fractions above the staff give contact points that
specify which point along the bow is to touch the string at which time. BCPs are given as
fractions between 0 and 1, with 0 indicating the talon and 1 indicating the point;
intermedial values indicate a point somewhere between. For example, the BCPs $\frac{0}{4}
\longrightarrow \frac{1}{4} \longrightarrow \frac{2}{4}$ indicate a downbow drawn
smoothly from the talon ($\frac{0}{4}$), to a point one quarter of the way up the bow
($\frac{1}{4}$), and then to exactly the midway point of the bow ($\frac{2}{4}$); the
BCPs $\frac{6}{7} \longrightarrow \frac{7}{7} \longrightarrow \frac{0}{7}$ indicate a
downbow drawn smoothly from a point $\frac{6}{7}$ of the way up the bow, up to the point
of the bow ($\frac{7}{7}$), and then a full upbow drawn all the way back to the talon
($\frac{0}{7}$). The BCPs used in the piece all divide the bow evenly into four
($\frac{0}{4}, \frac{1}{4}, \frac{2}{4}, \frac{3}{4}, \frac{4}{4}$) or seven
($\frac{0}{7}, \frac{1}{7}, \frac{2}{7}, \frac{3}{7}, \frac{4}{7}, \frac{5}{7},
\frac{6}{7}, \frac{7}{7}$) parts; the BCPs $\frac{0}{4}$ and $\frac{0}{7}$ are synonyms
for the talon, and the BCPs $\frac{4}{4}$ and $\frac{7}{7}$ are synonyms for the point.
The bow contact points in \textit{Stirrings Still} animate pitched lines (frequently
glissandi) in two ways at the same time: first, with structured --- and sometimes quite
fast --- changes of bow; and, second, with frequently surprising changes of bow color,
from scratch to flautando and back again. This happens because all the BCPs in the score
are rhythmed: transitions between contact points drive changes in the \textit{speed} of
the bow along its course. This means that overly exacting interpretations of, say,
$\frac{5}{6}$ versus $\frac{6}{6}$ aren't necessary to bring out the constant changes of
color the notation invites. (It may also be worth noting that almost all of the contact
points specified in the score come from a repeating pattern of 15 bow changes.)

\textbf{Other articulations.} The symbol $\dashv$ means ``stop-on-string'': deaden the
string with the bow at the end the notated duration; no string resonance should be heard.

\textbf{Fermatas.} The score notates four types of fermata: short, medium, long, very
long. These are the same as \textit{Akasha} and should be interpreted as in the previous
piece: the short fermata should be surprisingly short, suggesting the time of only a
fraction of beat; the medium fermata should be interpreted as a usual fermata; the long
fermata should last long enough that our memory begins to retrace what was last heard;
and the very long fermata should last long enough that we \textit{almost} begin to forget
the previous flow of musical time. (Fermatas over the narrator's text document decisions
about the distribution of the text but aren't to be performed.)

\textbf{Circular bowing.} The piece makes extensive use of circular bowing, always
introduced by an arrowed circle and a directive giving the width (or degree of stricture)
of the circled stroke: \textit{tight}, \textit{very tight}, \textit{wide}, \textit{very
wide}, and so on. (The location of the stroke on the string --- whether closer to the
bridge or the nut --- is left to the players; the priority is a kaleidoscopic variegation
of color.) \textbf{Stems indicate the rate at which circular strokes repeat.} So, the
first appearance of the technique, at measure 23, specifies four (moderately wide)
circular strokes in the first violin, against three in the second violin, against five in
the viola. The combined ``circular polyrhythm'' is crucial to convey: think the turning
of gears in a fantastical machine that we hear but do not see. Note that these rhythms
layer different bowspeed \textit{colors} on top of one another, rather than different
accent chains as in conventional polyrhythm: because the viola's five circles are faster
than the violins' (and because all three instruments are playing circles of roughly the
same width) the viola's color is lighter and faster (and closer to a type of circular
flautando) than of either of the violins'. (Resulting differences of color are more
pronounced in the cello versus the other instruments: the cello's \textit{wide} stroke
travels a greater distance than the other instruments', giving the cello a ``faster
color'' than the other instruments at the same width.)

\textbf{Glissandi.} Do not rearticulate note-heads in the middle of glissandi.
